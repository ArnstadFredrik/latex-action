\setstretch{1}
\begin{scriptsize}

\noindent
\textbf{Eksempler på fusk i forbindelse med oppgaver og hjemmeeksamener:}
\begin{itemize}[leftmargin=0.5cm]
\item gjengivelse av stoff/materiale hentet fra lærebøker, andre fagbøker, tidsskrifter, egne eller andres oppgaver osv. som er framstilt uten kildehenvisning og klar markering av at det er sitater 
\item besvarelse eller tekst som er hentet fra internett og utgitt som egen besvarelse 
\item besvarelse som i sin helhet er brukt av studenten ved en tidligere eksamen 
\item besvarelse som er brukt av en annen person ved en tidligere eksamen 
\item besvarelse som er utarbeidet av en annen person for studenten 
\item innlevert arbeid av praktisk eller kunstnerisk art som er laget av andre enn studenten selv 
\item samarbeid som fører til at en besvarelse i det alt vesentlige er lik en annen besvarelse til samme eksamen der det kreves individuelle besvarelser
\end{itemize}

\noindent
\textbf{Retningslinjer om fusk finner du her:}
\begin{itemize}[leftmargin=0.5cm]
\item lov 1. april 2005 nr. 15 om universiteter og høyskoler § 4-7 og § 4-8 
\item forskrift 11. desember 2015 nr. 1665 om opptak, studier, eksamen og grader ved VID vitenskapelige høgskole § 30
\item retningslinjer for behandling av fusk eller forsøk på fusk ved VID vitenskapelig høgskole fastsatt 18. desember 2015.
Ved å signere med navn erklærer jeg å være kjent med VID vitenskapelige høgskoles retningslinjer om plagiering og fusk, og at min besvarelse er i samsvar med disse bestemmelsene.
\end{itemize}
\end{scriptsize}

\begin{center}
	\vspace*{\fill}
	\vspace*{\fill}
\end{center}
